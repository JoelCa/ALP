\documentclass[a4paper,11pt]{article}
\usepackage[left=1.5cm,top=2.0cm,right=1.5cm,bottom=2.0cm]{geometry}
\usepackage[T1]{fontenc}
\usepackage[utf8]{inputenc}
\usepackage{lmodern}
\usepackage[spanish]{babel}
\usepackage{hyperref}
\usepackage{graphicx}
\usepackage{caption}
\usepackage{subcaption}
\usepackage{amssymb}
\usepackage{tikz}
\usetikzlibrary{positioning, babel}
\newdimen\nodeDist
\nodeDist=35mm

\newcommand\tab[1][0.5cm]{\hspace*{#1}}

\title{}
\author{}

\begin{document}

\maketitle

\begin{center}
\section*{Análisis de lenguajes de programación} 
\end{center}

\newpage{\pagestyle{empty}\cleardoublepage}

\section{Data types}

\begin{verbatim}
data ProofState = PState { position :: [Int]
                         , context :: [Context]
                         , typeContext :: [TypeContext]
                         , ty :: [Maybe (Type, TType)]
                         , term :: [SpecialTerm]
                         , subp :: Int
                         , name :: String
                         , tyFromCut :: [Type]
                         , subplevel :: [Int]
                         }
\end{verbatim}

\begin{itemize}
  \item position: cantidad de variables de términos ligadas, para cada prueba.
  \item context: contexto de variables de términos, para cada prueba.
  \item typeContext: contexto de variables de tipo, para cada prueba.
  \item ty: tipos objetivos de las pruebas no concluidas.
  \item term: representa al lambda término, el cual se va construyendo por partes.
  \item subp: cantidad total de pruebas no terminadas, o activas.
  \item name: ?.
  \item tyFromCut: ?.
  \item subplevel : Indica la cantidad de sub-pruebas que están activas por nivel.
\end{itemize}


\begin{verbatim}
data SpecialTerm = HoleT (Term->Term) | DoubleHoleT (Term->Term->Term) |
                   Term Term | TypeH TypeHole

data TypeHole = HTe ((Type, TType) -> Term) | HTy ((Type, TType) -> TypeHole)
\end{verbatim}

El data type \texttt{SpecialTerm} representa a un lamba término que puede tener dos tipos de agujero: lambda términos o tipos, de manera
exclusiva, es decir solo puede tener un solo tipo de agujero.

El data type \texttt{TypeHole} representa agujeros de tipos en un lambda término.


\section{Observaciones}
\begin{itemize}
  \item El comando \texttt{apply} puede actuar como el comando \texttt{assumption}.
  \item Cuando aplicamos el comando \texttt{intro} sobre un goal con ``para todo'', las variables
  ligadas al/los para todo/s, pasan a ser libres en el nuevo goal. 
  \item La substitución que encuentra el algoritmo de unificación es un termino cerrado.
  \item Sean $\Gamma_{t}$ el contexto de tipos inicial, y $\alpha$ un tipo dado por el usuario. Luego, $FV(\alpha) \subseteq \Gamma_{t}$.
  Además, si $FV(\alpha) \cap BV(\alpha) \neq  \emptyset$, se encuentra mediante substituciones de variables de tipos ligadas, un tipo $\alpha'$ 
  equivalente a $\alpha$, tal que $FV(\alpha') \cap BV(\alpha') = \emptyset$, y $FV(\alpha') = FV(\alpha)$.
  
  El contexto de tipos inicial, se define a partir del comando \texttt{Props}.
  
  \item Al implementar el comando \texttt{intro} sobre un ``para todo'', no es necesario chequear que no la variable de tipo ligada no se encuentre
  en el contexto de tipos, pues ya sabemos que no está allí, gracias al renombramiento de variables de tipos ligadas.
  
  \item Construimos un lambda término $t$, tal que $t:\alpha$, donde $\alpha$ es el tipo (asociado a una fórmula lógica), dado por el usuario. El isomorfismo
  de Curry-Howard, nos permite relacionar a los tipos con fórmulas lógicas, y a las pruebas con lambda términos.
  El lambda término $t$, se construye a partir de la prueba que da el usuario.
  El teorema de Curry-Howard, nos asegura que, si $t:\alpha$, entonces la fórmula asociada a $\alpha$ es válida.
  
  \item La función \texttt{habitar} se encarga de construir el lambda término asociado a la prueba que va dando el usuario.
  A cada regla de la deducción natural del sistema IFC, le corresponde una regla de tipo, en el lambda cálculo simplemente
  tipado. ¿Como construimos el término $t$, asociado a una prueba lógica?.
  A cada aplicación de una regla de tipo, le corresponde un lambda término ``vació''.
  Es decir, si se busca un lambda término de tipo $\beta \rightarrow \gamma$, por ejemplo, y se lo quiere construir mediante la regla 
  de tipo \textbf{T-Abs}, entonces, tenemos el término vació, $t_{1} = \lambda x:\beta. \; ?$, donde $? : \gamma$.
  Representamos a los términos vacíos, mediante funciones, en el caso anterior, tendríamos la función:

  \begin{verse}
  $t_{1}$ :: \texttt{Term} $\rightarrow$  \texttt{Term} \\
  $t_{1}$ $t$ = $\lambda x:\beta. \; t$
  \end{verse}

  Donde, \texttt{Term} es el tipo de los lambda términos. El argumento que se le pase a $t_{1}$, debería tener tipo $\gamma$.
  Esto se chequea de acuerdo a qué regla de tipo quiere aplicar el usuario.
  Por lo tanto, $t = (t_{1} . t_{2} \dots t_{n-2} . t_{n-1}) \; t_{n}$, donde $t_{n}$ es un lambda término no vació, que sirve de argumento a la composiciones
  de funciones. Cada $t_{i}$, con $i \in \{ 1 \dots n \}$, se corresponde con una regla de tipado, que a su vez se corresponde con una regla de deducción natural, 
  que da el usuario.
  
  \item Utilizamos el siguiente AST para representar a un lambda término:
  \begin{verbatim}
  data Term  = Bound Int
             | Free Name 
             | Term :@: Term
             | Lam (Type,TType) Term
             | BLam Var Term
             | Term :!: (Type,TType)
  \end{verbatim}

  Donde, \texttt{Type} es el AST de los tipos, y \texttt{TType} es el AST de los tipos sin nombres, es decir que estos emplean la notación De Bruijn.
  \texttt{Type} nos es útil a la hora de imprimir en pantalla el tipo del lambda término, de modo tal que tenga el tipo ingresado por el usuario.
  Mientras que, \texttt{TType} lo usamos en todas las operatorias sobre los tipos de los lambda términos, por ejemplo, en la unificación de tipos, que
  se realiza cuando se aplica la regla de eliminación del \textbf{para todo}.
  
  Si solo emplearíamos \texttt{TType}, el tipo de lambda término no coincidiría con el tipo dado por el usuario.
  
  \item Cuando el tipo $\alpha$ no contiene \textbf{para todos} ni \textbf{existes}, la representación de $\alpha$ mediante el AST \texttt{Type} es 
  prácticamente la misma que \texttt{TType}, solo varían en el nombre de los constructores.
  
  \item Cuando creamos sub-pruebas es necesario mantener una copia de los siguientes atributos de la prueba actual.
  \begin{itemize}
    \item position.
    \item context.
    \item typeContext.
  \end{itemize}
   Una vez que termina una prueba, la siguiente puede continuar a partir del estado de los atributos copiados.
   La idea del algoritmo que implementamos es la siguiente. Cuando un usuario crea $n$ pruebas en un estado $s_{0}$, donde $n > 1$, hacemos en total $n-1$ copias 
   de $s_{0}$, una copia cada vez que termina una sub-prueba. Consideramos a un estado como una instancia de los tres atributos dados anteriormente.
   Sean $s$ una lista de estados, $s_{i}$ un elemento de $s$, y tanto $spt$, como $spl$, listas de enteros. Donde, $spt$ y $spl$ se corresponden a los campos 
   \texttt{subp} y \texttt{subplevel} del registro \texttt{ProofState}.
   Podemos ver en la Figura \ref{Estados}, como se aplica el algoritmo para un caso específico. 
   
  \begin{figure} 
  \begin{tikzpicture}[node distance=2cm] 
  \path 
    (0,0) node(A)[align=center,   below] {$n = 0$ \\ $s = [ x_{0} ]$ \\ $spt = 1$ \\ $spl = [1]$}
    (3,0) node(B)[align=center, below] {$n = 2$ \\ $s = [ x_{0}, x_{0} ]$ \\ $spt = 2$ \\ $spl = [2]$}
    (6,0) node(C)[align=center,  below] {$n = 0$ \\ $s = [ x_{1}, x_{0} ]$ \\ $spt = 2$ \\ $spl = [2]$}
    (9,0) node(D)[align=center,  below] {$n = 2$ \\ $s = [ x_{1}, x_{1}, x_{0} ]$ \\ $spt = 3$ \\ $spl = [2,2]$}
    (12,0) node(E)[align=center,  below] {$n = 0$ \\ $s = [ x_{2}, x_{1}, x_{0} ]$ \\ $spt = 3$ \\ $spl = [2,2]$}
    (15,0) node(F)[align=center,  below] {$n = -1$ \\ $s = [ x_{1}, x_{0} ]$ \\ $spt = 2$ \\ $spl = [1,2]$}
    (15,-3) node(G)[align=center,  below] {$n = 0$ \\ $s = [ x_{3}, x_{0} ]$ \\ $spt = 2$ \\ $spl = [1,2]$}
    (12,-3) node(H)[align=center,  below] {$n = -1$ \\ $s = [ x_{0} ]$ \\ $spt = 1$ \\ $spl = [1]$}
    (9,-3) node(I)[align=center,  below] {$n = 0$ \\ $s = [ x_{4} ]$ \\ $spt = 1$ \\ $spl = [1]$}
    (6,-3) node(J)[align=center,  below] {$n = -1$ \\ $s = [ \; ]$ \\ $spt = 0$ \\ $spl = [ \; ]$};
  
  \draw[->]
    (A) edge (B) (B) edge (C) (C) edge (D) (D) edge (E)  (E) edge (F) (F) edge (G)
    (G) edge (H) (H) edge (I) (I) edge (J);
 
  \end{tikzpicture}
  \caption{Diagrama de estados.} \label{Estados}
  \end{figure}
  
  \item Aunque halla un lambda término con variables de tipo que tengan el mismo nombre que las variables de términos, su semántica de tipos está
  bien determinada.
  
  \item Al hacer el parser del lambda cálculo, asumí que la aplicación de tipos tiene más precedencia que la aplicación de términos, ¿está bien?.
  Ejemplo: $x \; y \; [A] = x \; (y \; [A]).$

\end{itemize}

\section{Lo que falta}
\begin{itemize}
  \item Arreglar el parser, respecto a los operadores que define el usuario (precedencia y demás).
  Ver caso: \texttt{Definition x x y := forall w, forall v, x and y}.
  \item Hacer que el lambda término devuelto en la construcción de una prueba, puede ser inferido por el comando \texttt{infer}.
  Ver caso: \texttt{Theorem: p -> q -> (((p and q) and (p and p)) and q) and p}.
\end{itemize}












\end{document}
